% !TEX encoding = UTF-8 Unicode
% !TEX TS-program = pdflatex

\chapter{Introduzione}

Questa guida si rivolge sia a chi desidera mettere in funzione \name, sia a chi vuole intraprendere nuovi sviluppi del sistema, per questi ultimi, prima di intraprendere lo studio della presente guida, è fortemente consigliata la lettura di alcune guide di base sull'uso di Arduino, sui protocolli di comunicazione seriale e in particolare l'I$^2$C, lo SPI e la comunicazione seriale UART (sulla quale si deve prestare molta attenzione alla questione byte-ASCII). Infine può tornare utile la conoscenza del protocollo OneWire.

Di fondamentale importanza è avere una basilare esperienza di programmazione e possibilmente un po' di conoscenza del linguaggio C.

Infine è desiderabile, per tutti i lettori, conoscere il modo in cui determinati sensori funzionano, benchè molte feature siano indicate nei capitoli dedicati è sicuramente fondamentale far riferimento ai datasheet per comprenderne meglio le caratteristiche. In particolare è utile conoscere il funzionamento di accelerometri, giroscopi e magnetometri MEMS, come interagire con essi e come si può ricavare l'assetto dai dati 'raw' ottenuti.

Per gli aspiranti alla sola utilizzazione del sistema si consiglia una lettura dei soli capitoli relativi alla messa in funzione dello strumento e all'interpretazione dei dati. Il capitolo su \software darà le istruzioni sull'interfaccia grafica appositamente studiata per interagire con \name per la lettura dei dati in tempo reale e per l'invio di determinati comandi e settaggi.
Questo software, scritto in Python2.7, è in grado anche di leggere i file *.txt salvati su SD dal datalogger.

Per una trattazione sommaria di alcuni argomenti utili si rimanda alle appendici, mentre per quanto concerne lo studio approfondito del filtro utilizzato per ricavare l'assetto si rimanda alle seguenti fonti:

\begin{itemize}
\item Sebastian O.H. Madgwick. An efficient orientation filter for inertial and inertial/magnetic sensor arrays. Rapp. tecn. Bristol University 2010.
\item Sebastian O.H. Madgwick. http://www.c-io.co.uk/open-source-imu-and-ahrs-algorithms
\item R. Vaidyanathan, Sebastian O.H. Madgwick, A.J.L. Harrison. Estimation of IMU and MARG orientation using a gradient descent algorithm. In:Rehabilitation Robotics (ICORR), 2011 IEEE International Conference on (2011), pp 1-7.
\end{itemize}