% !TEX encoding = UTF-8 Unicode
% !TEX TS-program = pdflatex

\chapter{Invio e ricezione dati}
La sintassi generale per l'invio e la ricezione di dati, siano essi comandi o dati rilevati, è la seguente:
\begin{lstlisting}%
	[firstline=1, lastline=3, caption={sintassi per lo scambio seriale di informazioni}, label={lst:stringaFPV}]
	$TYPE,val_1,val_2,...,val_n*checksum\n
\end{lstlisting}
dove \$ è il carattere di inizio stringa, TYPE è un byte che identifica il tipo di stringa che stiamo inviando, dopo la virgola invece vengono inseriti i comandi sotto forma di byte anch'essi suddivisi da virgole. Siamo in grado di inviare un byte alla volta, quindi tutti i formati vanno suddivisi in bytes e inviati un pezzo alla volta, quindi va prevista una ricomposizione da parte del ricevente.

\'E evidente che si tratti di un formato del tutto simile alle stringhe NMEA, con un'aggiunta importante: in questo caso, dato che vogliamo scambiare solo valori numerici, aumentiamo le prestazioni inviando direttamente il valore sotto forma di byte e non sotto forma di carattere.\newline
Si prenda ad esempio il valore $255$, il massimo inviabile con un solo byte, se lo inviassimo sotto forma di caratteri dovremmo inviare tre bytes, facendolo sotto forma di byte riduciamo il peso dell'invio e aumentiamo la velocità.

Si deve però prestare attenzione al tipo di device che abbiamo di fronte, questo è possibile farlo solo qualora si abbia il controllo su entrambi i lati dello scambio di informazioni. Qualora si abbia a che fare con un GPS per fare un esempio, dobbiamo uniformarci al suo protocollo che invece invia i caratteri singolarmente e quindi per interpretarli in modo corretto è necessario conoscere la codifica ASCII, un approfondimento è presente nelle appendici della guida.

\section{FPV Telemetry}
Lo scambio di informazioni in telemetria è reso possibile attraverso il modulo FPV che lavora a $533MHz$. Lato \name è collegato ad una porta seriale (di default alla numero 3), lato computer invece si collega alla porta USB in quanto il modulo di terra è dotato di un convertitore USB-UART. La seriale lavora a $57600 baud$.

Proprio per questo modulo abbiamo previsto il sistema di comunicazione indicato nel listato \{\ref{lst:stringaFPV}\}.

Al momento il sistema è predisposto per la ricezione di comandi composti da un byte di type e uno di comando, per modificare la lunghezza è sufficiente andare a cambiare il valore della label LENGTH presente nel file Const.h allegato al codice sorgente, il valore li presente tiene conto solo dei byte di comando e non quelli di tipo.

