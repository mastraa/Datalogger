% !TEX encoding = UTF-8 Unicode
% !TEX TS-program = pdflatex

\chapter{Obiettivi}

Si è voluto progettare e costruire un apparato elettronico utile per acquisire molteplici informazioni sulle imbarcazioni durante la navigazione. L'utilizzo di questi dati sarà poi da un lato finalizzato al miglioramento tecnico degli equipaggi, dall'altro servirà per affinare le tecniche di progettazione per valutare se effettivamente le barche rispondo secondo quanto voluto in fase di progettazione. Infine, in funzione di un progetto molto più ampio, le rilevazioni potranno essere utili come feedback per l'affinamento delle tecniche di simulazione CFD.

L'interesse nel produrre un sistema a basso costo e completamente customizzabile ci hanno indirizzato verso il mondo Arduino che, almeno in un primo momento, sarà il cervello del nostro apparato. Si prevede in futuro di ampliare il sistema e di conseguenza potrebbe essere utile l'inserimento di sistemi più potenti come ad esempio Raspberry abbinati a microcontrollori eventualmente standalone. La scarsa conoscenza ed esperienza in elettronica e in programmazione ci hanno fatto optare per l'utilizzo di schede preassemblate con una buona quantità di documentazione disponibile


Le funzionalità richieste sono:
\begin{itemize}
\item rilevamento dell'assetto;
\item rilevamento della posizione;
\item rilevamento del vettore velocità (modulo e verso);
\item rilevamento delle condizioni ambientali (vettore veloicità del vento, temperatura);
\item predisposizione per la misura di carichi e deformazioni;
\item salvataggio dati;
\item invio dati in tempo reale (successiva implementazione);
\item schermo per la visualizzazione id informazioni utili;
\end{itemize}